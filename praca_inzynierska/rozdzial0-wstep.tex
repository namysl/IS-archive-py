\chapter*{Wstęp}
\addcontentsline{toc}{chapter}{Wstęp}

W przeciągu kilku ostatnich lat Kubernetes stał się jednym z najbardziej popularnych wolnoźródłowych rozwiązań zarządzania infrastrukturą w projektach informatycznych. Wykorzystanie tej platformy usprawnia proces szybkiego dostarczania skonteneryzowanych aplikacji, a ponadto ułatwia i automatyzuje część zadań związanych z ich administracją. 

W niniejszej pracy stworzono dwa rodzaje klastrów Kubernetesa oraz przeprowadzono testy, służące oszacowaniu ich wydajności. W tym celu wykorzystano narzędzia z otwartym źródłem, które umożliwiają konteneryzację aplikacji, automatyzację powtarzalnych zadań, monitoring oraz testowanie stworzonej infrastruktury. Ponadto opisano teoretyczne aspekty związane z wirtualizacją, konteneryzacją oraz orkiestracją kontenerami. 
